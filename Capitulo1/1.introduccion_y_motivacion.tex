

\chapter{Introducción}


La optimización multiobjetivo es una disciplina dentro de las matemáticas aplicadas, con sus orígenes remontándose al siglo XIX. Esta área se destaca por abordar la compleja pregunta formulada por un economista de esa época, quien se interrogaba sobre la forma óptima en que podría vivir un grupo de personas. Esta pregunta llevó al desarrollo de una de las definiciones más cruciales en este campo: \textbf{el frente de Pareto}. La idea central del frente de Pareto es que la situación ideal en un sistema de recursos limitados no es aquella donde todos maximizan sus recursos, sino más bien aquella en la que ninguna mejora individual es posible sin perjudicar a otro. 

En el intento de comprender y abordar las complejidades del mundo, a menudo tendemos a simplificar los problemas, buscando respuestas concretas en las que se optimiza un solo objetivo aunque esta manera de verlo no caracterice de manera fiel a los problemas. El campo de la optimización multiobjetivo se desvía de esta tendencia, cambiando su enfoque a estudiar problemas con múltiples dimensiones en sus espacios de objetivos. Estos problemas ya no requieren soluciones puntuales, sino conjuntos de soluciones. Dichos conjuntos de soluciones son tales que no existe una forma en la que todos los aspectos del problema puedan ser mejorados simultáneamente. Con el tiempo, la relevancia y aplicabilidad de la optimización multiobjetivo han crecido mucho, encontrando uso en áreas tan diversas como la bioinformática \cite{handlMultiobjectiveOptimizationBioinformatics2007}, redes inalámbricas \cite{gunjanReviewMultiobjectiveOptimization2023}, procesamiento del lenguaje natural (NLP) \cite{sainiMultiobjectiveOptimizationTechniques2021}, procesamiento de imágenes \cite{aslamComprehensiveSurveyOptimization2020}, y en los campos de la astronomía y astrofísica \cite{mullerUsingMultiobjectiveOptimization2023}. A diferencia de los problemas de un solo objetivo, donde técnicas como el descenso del gradiente han demostrado ser exitosas en aprendizaje automático y redes neuronales, los problemas multiobjetivo presentan una complejidad adicional que exploraremos a lo largo de este trabajo.


Al abordar problemas multiobjetivo, nos encontramos con desafíos únicos. Uno de los más significativos es la imposibilidad de establecer un orden total en un espacio con múltiples objetivos.  Este obstáculo se ve exacerbado por la \emph{maldición de la dimensionalidad}, un fenómeno bien conocido en el campo de la computación, donde el aumento de las dimensiones de un problema no resulta en una simple generalización de un problema unidimensional. La complejidad aumenta de manera no trivial con cada dimensión adicional, lo que hace que muchas técnicas de optimización convencionales sean inadecuadas para estos problemas.

Frente a esta complejidad, un enfoque que ha demostrado ser especialmente valioso es el uso de \textbf{algoritmos evolutivos} \cite{coelloEvolutionaryAlgorithmsSolving}. Inspirados en los procesos evolutivos de la naturaleza, estos algoritmos buscan encontrar estrategias exitosas para producir soluciones altamente adaptables y diversas. Aplicados al área de Optimización Multiobjetivo, estos algoritmos, conocidos como Algoritmos Evolutivos para Optimización Multiobjetivo (EMOAs o MOEAs por sus siglas en inglés), se caracterizan por su capacidad para generar progresivamente una población más apta a través de un método iterativo. Una ventaja clave de los EMOAs es su habilidad para operar sin necesidad de conocer la forma cerrada de las funciones objetivo, una situación común en los problemas multiobjetivo. En este trabajo, exploraremos en profundidad el potencial y la aplicación de los algoritmos evolutivos en la resolución de problemas multiobjetivos complejos.

El campo de los EMOAs  es vasto y diverso, con varias clasificaciones para diferentes enfoques en la aproximación a los Problemas de Optimización Multiobjetivo (MOP). Un enfoque particularmente interesante es el de los algoritmos basados en \textbf{indicadores de calidad}, que se detallarán en la Sección \ref{sec:SMS-EMOA}. Estos algoritmos utilizan características específicas, o indicadores, de las soluciones propuestas para guiar el proceso de búsqueda evolutiva con el fin de encontrar la mejor representación del frente de Pareto posible. Una clasificación que resultará útil de estos indicadores es la manera en la que caracterizan al conjunto de soluciones. En particular, nos concentraremos en dos categorías importantes que son los indicadores de convergencia y los indicadores de diversidad. Los indicadores de convergencia evalúan qué tan cerca se encuentra el conjunto de soluciones del frente de Pareto, o de una aproximación de este. Por otro lado, los indicadores de diversidad examinan la uniformidad y el espaciamiento entre las soluciones. Mejorando ambos tipos de indicadores—uno de convergencia y otro de diversidad—se logra una representación más fidedigna del frente de Pareto (ver Figura \ref{fig:aproximacion}).

A pesar de la existencia de numerosos esfuerzos dirigidos a resolver problemas multiobjetivo mediante el diseño de nuevos algoritmos evolutivos, este trabajo se centra en un enfoque diferente. En lugar de desarrollar un nuevo método, nos dedicaremos al estudio exhaustivo de diferentes configuraciones de un algoritmo evolutivo ya establecido; el EMOA llamado PFI-EMOA (Pareto Front Invariant-EMOA) \cite{PFI}. Este algoritmo realiza una combinación ponderada entre un indicador de convergencia conocido como IGD+ \cite{IGD} y un indicador de diversidad denominado Energía-S de Riez \cite{sEnergy} para guiar la búsqueda en el espacio de estados y obtener conjuntos de soluciones que no tengan problemas encontrando frentes de Pareto con los que otros algoritmos han tenido dificultades. Las diferentes configuraciones que probaremos corresponden a cambiar el peso con el que se combinan estos indicadores en el algoritmo. 
    
Dado lo anterior, podemos resumir las intenciones de este trabajo en las siguientes preguntas:

\begin{itemize}
    \item La primera y más fundamental es: \textbf{¿Existe una diferencia en el desempeño del algoritmo PFI-EMOA al variar la combinación de IGD+ y Energía-S?}
    La respuesta a esta pregunta no es obvia, ya que podría ser que la combinación específica de indicadores no importe siempre y cuando se usen ambos para atacar el problema. Para probar este punto se usaron un conjunto de problemas de prueba que ya son estándar en la literatura de MOPs para vigilar el desempeño en cada uno de ellos. 
    \\Encontramos que si existe una diferencia significativa para la mayoría de los problemas. Es decir, en algunos casos, si importa si escogemos darle mayor contribución a un operador de convergencia que a alguno de diversidad. Mientras que en otros, es mejor darle mayor peso al indicador de diversidad.
    \item Después sigue la pregunta natural: \textbf{¿Qué combinación de pesos es mejor para cada familia de problemas?}
    De igual manera esta respuesta no será obvia. Habrá problemas que por su forma requieran más exploración del espacio de estados y esto puede significar que necesitan tener una contribución mayor del indicador de diversidad. \\
    Para cada problema esto será distinto, sin embargo, encontramos resultados generales para agrupaciones útiles como por el número de objetivos y por el tipo de indicadores de calidad. De forma general encontramos que darle más peso a IGD+ resulta ser mejor para obtener buenos resultados de convergencia y viceversa para Energía-S, teniendo el caso excepcional del mismo Energía-S. Es decir, el intentar optimizar este indicador durante la búsqueda, no asegura que obtengamos soluciones que lo maximicen.
    \item Como también variamos la combinación de indicadores con respecto a PFI-EMOA la siguiente pregunta es: \textbf{¿Qué sucede cuando cambiamos el indicador de IGD+ a otro indicador de convergencia?}
    En este trabajo, además de estudiar el efecto de IGD+ a detalle, sustituimos el indicador de convergencia R2 \cite{R2} que se explica en la sección \ref{sec:R2}. Aunque ambos indicadores se preocupan principalmente con la convergencia del conjunto de soluciones, se definen de manera distinta, por lo que no tenemos garantizado obtener el mismo resultado al usar uno o el otro.\\
    Encontramos que la decisión de darle diferentes pesos a la combinación de indicadores cuando está presente el IGD+ es mucho más importante que la del R2. Teniendo este último muy pocas situaciones donde existe una diferencia significativa entre una y otra combinación de pesos.
   
\end{itemize}


Para abordar las preguntas planteadas, llevamos a cabo un protocolo de experimentación empírico, utilizando distintas combinaciones de los indicadores IGD+ y Energía-S. Aquí surge la pregunta de cómo vamos a evaluar el desempeño de cada una de las diferentes combinaciones. Para esto necesitamos evaluar los algoritmos bajo las mismas circunstancias. Así, por cada diferente configuración de hiperparámetros (pesos de los indicadores de convergencia y distribución) primero se toma un conjunto de problemas prueba con diferente número de objetivos (como se puede ver en la tabla \ref{Tabla:problema_prueba}) y después se evalúa la solución obtenida usando indicadores de calidad que incluyen: Hipervolumen \ref{sec:HV}, Distancia Generacional Invertida IGD \ref{sec:IGD}, Epsilon +\ref{sec:Epsilonp}, Diversidad de Solow-Polasky \ref{sec:SPD}, R2 \ref{sec:R2} y los dos que se usaron para guiar la búsqueda; es decir, IGD+ y Energía-S de Riesz. Entonces, se obtiene una solución guiada por una combinación específica del indicador de convergencia y distribución y luego se evalúa usando otro indicador. Estas evaluaciones se realizan para poblaciones inicializadas de acuerdo a 10 semillas distintas. Así se obtienen 10 resultados por cada uno de los indicadores de evaluación y podemos compararlos entre sí usando pruebas estadísticas para decidir si una configuración de hiperparámetros es mejor a otra.

El objetivo del trabajo es determinar si ciertas combinaciones de pesos son consistentemente más efectivas que otras y si estos resultados varían según el problema específico. Inicialmente, analizamos el número de problemas en los que la elección de pesos tenía un impacto relevante. Posteriormente, nos centramos en identificar, para cada indicador y número de objetivos, las combinaciones de pesos que ofrecían un mejor rendimiento.

Este trabajo está organizado de la siguiente manera para facilitar una comprensión integral del estudio realizado:

\begin{itemize}
    \item \textbf{Capítulo 2: Fundamentos Previos.} Este capítulo proporciona una introducción detallada a los problemas multiobjetivo, estableciendo el lenguaje y los conceptos clave que serán utilizados a lo largo de la tesis.
    \item \textbf{Capítulo 3: Algoritmos Evolutivos para problemas Multiobjetivo.} Aquí definimos qué son los algoritmos evolutivos y cómo se aplican en la solución de problemas multiobjetivo. Se incluye una descripción de varios algoritmos evolutivos multiobjetivo exitosos, así como la posición del algoritmo que es el foco de este trabajo dentro de este contexto. También se presentan las definiciones de todos los indicadores de calidad que emplearemos, junto con la metodología para comparar los resultados de los algoritmos, utilizando pruebas estadísticas como Friedman y Wilcoxon.
    \item \textbf{Capítulo 4: Diseño Experimental.} En este capítulo, detallamos cómo se diseñó el procedimiento experimental, incluyendo la selección de algoritmos, las combinaciones de pesos utilizadas, los problemas de prueba seleccionados y el número de objetivos considerados. Se presentan algunos de los resultados más interesantes que forman la base para la discusión en el siguiente capítulo.
    \item \textbf{Capítulo 5: Discusión de Resultados.} Aquí discutimos los resultados obtenidos, enmarcándolos en el contexto de las preguntas de investigación planteadas anteriormente.
    \item \textbf{Capítulo 6: Conclusiones y Trabajo Futuro.} Finalmente, en el último capítulo, ofrecemos las conclusiones derivadas de nuestra investigación y delineamos posibles direcciones para futuros estudios en este campo.
\end{itemize}





