
% Thesis Abstract -----------------------------------------------------


%\begin{abstractslong}    %uncommenting this line, gives a different abstract heading
\begin{abstracts}        %this creates the heading for the abstract page

% This is where you write your abstract ...
% \blindtext

La optimización multiobjetivo es una rama de las matemáticas aplicadas, con orígenes que se remontan al siglo XIX. Este campo se destaca por abordar preguntas complejas relacionadas con buscar una decisión óptima en un problema con varias características a optimizar. Esta complejidad en el planteamiento del problema llevó al desarrollo del concepto del frente de Pareto. Este concepto subraya que la situación ideal en un problema multiobjetivo no será necesariamente donde todos los  objetivos se optimizan individualmente, sino más bien un conjunto de soluciones en las que no se pueda encontrar una solución que sea mejor que las demás en todos los objetivos.

La investigación de este trabajo se enfoca en el uso de algoritmos evolutivos para la optimización multiobjetivo, conocidos como MOEAs. Estos algoritmos se inspiran en los procesos evolutivos de la naturaleza para producir soluciones adaptables y diversas, operando sin necesidad de conocer la forma exacta de la función objetivo. La tesis se centra en estudiar de manera estadística diferentes hiperparámetros dentro de un MOEA existente, el algoritmo conocido como PFI-EMOA. El objetivo del trabajo es primero demostrar y luego comprender por qué ciertos parámetros resultan más eficaces que otros.

El estudio incluye una serie de experimentos utilizando combinaciones de los indicadores IGD+, R2 y Energía-S de Riesz en una variedad de problemas de prueba. A través de comparaciones estadísticas, se evaluó la existencia y magnitud de diferencias significativas en el rendimiento del algoritmo. Se buscó determinar si ciertas combinaciones de pesos eran consistentemente más efectivas que otras, o si los resultados variaban según el problema específico.

Mirando hacia el futuro, la tesis plantea preguntas para investigaciones subsecuentes. Una de ellas es qué tipos de problemas favorecen ciertas combinaciones de indicadores y la posibilidad de modificar dinámicamente los pesos de los indicadores en respuesta a los resultados intermedios del algoritmo. Esto permitiría una búsqueda más eficiente, inclinándose hacia el indicador de convergencia o diversidad según sea necesario.

\end{abstracts}
%\end{abstractlongs}


% ----------------------------------------------------------------------